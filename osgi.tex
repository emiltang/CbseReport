\section{OSGI}
Instead of using \emph{bundle activators} or \emph{constructors},
the \emph{OSGI declarative services} provide the annotations,
\mintinline{java}|@Activate| and \mintinline{java}|@Deactivate|.
These annotations can be used to control the behavior of services when they
are loaded and unloaded.
The annotations are used in the \emph{plugin services} to apply the plugins
resources to the \emph{game engine} when the plugin is loaded,
and then removing it again when the plugin is unloaded.

\begin{listing}
    \begin{minted}[frame=single]{java}
        @Reference(
        cardinality = ReferenceCardinality.MULTIPLE,
        policy = ReferencePolicy.DYNAMIC,
        policyOption = ReferencePolicyOption.GREEDY
        )
        private volatile List<IProcessor> plugins =
        new CopyOnWriteArrayList<>();
    \end{minted}
    \caption{The listing shows of how \texttt{IProcessor} services are being
    loaded into the core module.
    The collection of services is annotated with the
    \mintinline{java}|@Reference| annotation from OSGI declarative services.
    This annotation tells the OSGI framework to load all service
    implementations of the service interface into the collection.
    Additionally the annotation is configured to dynamically load services
while the program is running instead of only once.}
\label{lst:osgi-ref}
\end{listing}
